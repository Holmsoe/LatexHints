\documentclass{article}

\usepackage{graphicx}
\usepackage{amsmath,amsfonts,amssymb,amsthm}
\usepackage{array}
\usepackage{mathtools}
\usepackage[utf8]{inputenc}
\usepackage[danish]{babel}
\usepackage[T1]{fontenc}
\usepackage{fancyhdr}
\usepackage{multicol}
\usepackage{color}
\usepackage{booktabs}
\usepackage{subfig}
\usepackage{hyperref} %hyperref skal være den sidste pakke

\title{Fysiske enheder og SI systemet}
\author{Finn Andersen}
\date{\today}

%\pagestyle{fancy} 
%\pagenumbering{Roman}

\begin{document}

\maketitle
%\thispagestyle{empty} 

\begin{abstract}
Forståelsen af SI systemet og enhederne bag de fysiske størrelser gør det lettere at tilegne sig fysik. Ofte kan fysiske sammenhænge gættes ud fra kendskab til enhederne.
\end{abstract}

\begin{table}[h]
\caption{Oversigt over prefixer til SI enheder}
\begin{tabular}{llllllllll}
\toprule
deca-&	hecto-&	kilo-	&mega-&	giga-&	tera-	&peta-	&exa-	&zetta-	&yotta-\\
a&	h&	k&	M	&G	&T&	P&	E	&Z&	Y\\
$10^1$&	$10^2$	&$10^3$	&$10^6$	&$10^9$	&$10^{12}$	&$10^{15}$&	$10^{18}$&	$10^{21}$	&$10^{24}$\\\midrule									
deci-	&centi-	&milli-	&micro-&	nano-&	pico-	&femto-&	atto-	&zepto-&	yocto-\\
d&	c&	m&	$\mu$	&n	&p	&f	&a&	z&	y\\
$10^{-1}$&	$10^{-2}$	&$10^{-3}$	&$10^{-6}$	&$10^{-9}$	&$10^{-12}$	&$10^{-15}$&	$10^{-18}$&	$10^{-21}$	&$10^{-24}$\\\midrule
\end{tabular}
\end{table}

\section{Basisenheder}

\begin{table}[h]
\caption{Basisenheder}
\begin{tabular}{lcllc}
\toprule
Navn	&	Enhed	&	Hvad måles?	&	Symbol for det der måles	&	Dimension symbol\\\midrule
meter	&	m	&	længde	&	l (lille L), x, r	&	L\\
kilogram	&	kg	&	masse, (vægt)	&	m	&	M\\
sekund	&	s	&	tid	&	t	&	T\\
ampere	&	A	&	elektrisk strøm	&	I (stort i)	&	I\\
kelvin	&	K	&	temperatur	&	T	&	$\Theta$\\
candela	&	cd	&	lysintensitet	&	Iv (stort i med lille v som indeks)	&	J\\\midrule
\end{tabular}
\end{table}

\subsection{Længde.}
Der har i løbet af historien været mange længdemål. Typisk har man brugt nærliggende mål såsom længden af underarmen som grundlæggende længdeenhed. I 1795 opfandt franskmændene Delambre og Mechain et mere universelt længdemål. De definerede en meter som en 10 milliontedel af en fjerdedel af jordens omkreds. I dag ved vi, at jordens omkreds er 40 075 km, så de var tæt på. Det er denne definition der i dag danner grundlag for meterens størrelse. Blot er definitionen blevet ændret således at den er blevet mere nøjagtig. Man har siden 1799 haft den såkaldte normalmeter liggende i Paris som grundlag for definitionen. Fra 1983 er en meter defineret ud fra lysets hastighed i vacuum som den distance lyset passerer på \(\frac{1}{299,792,458}\) sekunder. I praksis anvendes følgende meterbetegnelser:

\begin{table}[h]
\caption{Anvendte længdemål}
\begin{tabular}{lllll}
\toprule
Navn &Symbol &Størrelse  &\\\midrule
kilometer&km&$10^3$&$1000$&meter\\
meter&m&$1$&1&meter\\
decimeter&dm&$10^{-1}$&$0.1$&meter\\
centimeter&cm&$10^{-2}$&$0.01$&meter\\	
millimeter&mm&$10^{-3}$&$0.001$&meter\\
mikrometer&$\mu$m&$10^{-6}$&$0.000001$&meter\\	%µm
nanometer&nm&$10^{-9}$&$0.000000001$&meter\\\midrule
\end{tabular}
\end{table}

Desuden anvendes i Danmark en række længdemål der ikke passer med SI systemet.

1 sømil = 
1 lysår =
1 ångstrøm =
1 tomme =

\subsection{Tid.}
Opdelingen af døgnet i 24 timer stammer helt tilbage til de gamle egyptere. Babylonerne anvendte en underopdeling i \(\frac{1}{60}\) omkring 300 BC. Omkring år 1000 ser man første gang opdelingen af døgnet i timer, minutter og sekunder hos den persiske videnskabsmand Al-Biruni. Denne opdeling var basis for definitionen af et sekund som \(\frac{1}{86,400}=\frac{1}{24 \times 60 \times 60}\) af et døgn. Sekund blev indført som basisenhed for tidsmåling i 1832. Definitionen baseret på Al-Biruni blev anvendt indtil 1960. Da døgnet eller jordens omløbstid om solen ikke er helt konstant besluttede man i 1960 erne at indføre en ny definition baseret på frekvensen af strålingen fra atomhenfald for et Caesium atom.\\
Normalt anvendes der ikke prefixer for tidsangivelser større end et sekund. For mindre tidsintervaller anvendes følgende tidsmål:

\begin{table}[h]
\caption{Anvendte tidsmål}
\begin{tabular}{lllll}
\toprule
Navn &Symbol &Størrelse  &\\\midrule
millisekund&ms&$10^{-3}$&$0.001$&sekund\\
mikrosekund&$\mu$s&$10^{-6}$&$0.000001$&sekund\\	%µm
nanosekund&ns&$10^{-9}$&$0.000000001$&sekund\\\midrule
\end{tabular}
\end{table}



\end{document}

