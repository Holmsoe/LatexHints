\documentclass{article}
\usepackage{graphicx}
\usepackage{amsthm,amsmath,amsfonts,amssymb}
\usepackage[utf8]{inputenc} %inputenc sørger for at der forstås æ, ø og å
\usepackage[danish]{babel} %kapitler etc. får danske navne
\usepackage[T1]{fontenc}

%pakker til tabeller
\usepackage{multirow}
\usepackage{booktabs} % giver mulighed for \toprule, \midrule og \bottomrule
%\usepackage[referable]{threeparttablex}
%\setTableNoteFont{\footnotesize}

%\renewcommand{\abstractname}{Opgaveformulering}%overskrift opgaveformulering
%\tableofcontents{arabic}
%\pagenumbering{arabic}

\begin{document}

\title{Demonstration af tabeller}
\author{Finn Andersen\\Parkvej 14\\Nykøbing F\\\texttt{fa@faxekalk.dk}}
\date{\today}
\maketitle

%\begin{abstract} 
%\end{abstract}

\section{Indledning}

Her er et simpelt eksempel med makebox.Vandrette linier er med for illustration. Denne metode kunne være nyttig ved simpel alignment.\\
%tabel med makebox
\newcommand{\tabelbox}[3]{\makebox[0.2\columnwidth][l]{#1} \makebox[0.2\columnwidth][c]{#2} \makebox[0.2\columnwidth][r]{#3}}
\makebox[0.63\columnwidth]{\hrulefill}\\
\tabelbox{Hej med dig}{din dummenik}{det er mig}\\
\makebox[0.63\textwidth]{\hrulefill}\\
\tabelbox{en}{linie}{mere}\\
\makebox[0.63\columnwidth]{\hrulefill}\\


%\begin{tabular}{l r r} %antal kolonner bestemmes af antal placeringsdefinitioner l,c,r

%\begin{tabular}{l |r || r|} %med lodrette linier. 
%PS. lodret linie fås med AltGr og tast med apostrof øverst til venstre.

\begin{tabular}{l|r|r}
& afstand & masse \\
\toprule
%\hline % vandret linie
Merkur  & 0.387 & 0.055 \\
Venus & 0.723 & 0.815 \\
\midrule
Jorden & 1.000 & 1.000 \\
Mars & 1.524 & 0.107 \\
\cline{2-2} %horisontal linie under søjle 2 til 2 = søjle 2
Jupiter & 5.203 & 317.83 \\
Saturn & 9.555 & 95.159 \\
\hline % vandret linie
\hline %dobbeltlinie
%bemærk, at lodrette linier brydes efter to vandrette.
Uranus & 19.218 & 14.500 \\
Neptun & 30.110 & 17.204 \\
Pluto & 39.545 & 0.0025 \\
\bottomrule
\end{tabular}

\section{Afsnit 1}

\begin{tabular}{l|r|r@{,}l} % @{,} betyder, at det automatiske mellemrum fjernes og erstattes af ,
%kan ikke bruges sammen med teksoverskrift
%\hline % vandret linie
Merkur  & 0.387 & 0&055 \\
Venus & 0.723 & 0&815 \\
Jorden & 1.000 & 1&000 \\
Mars & 1.524 & 0&107 \\
\cline{2-2} %horisontal linie under søjle 2 til 2 = søjle 2
Jupiter & 5.203 & 317&83 \\
Saturn & 9.555 & 95&159 \\
\hline % vandret linie
\hline %dobbeltlinie
%bemærk, at lodrette linier brydes efter to vandrette.
Uranus & 19.218 & 14&500 \\
Neptun & 30.110 & 17&204 \\
Pluto & 39.545 & 0&0025 \\
\hline % vandret linie
\hline %dobbeltlinie
\end{tabular}

\section{Afsnit 2}

\begin{center}
    \begin{tabular}{ | l | l | l | p{5cm} |} 
    % bemærk, at sidste søjle har en specificeret bredde og er defineret med p for tekstwrapping
    % uden denne vil teksten blot fortsætte.
    
    \hline
    Day & Min Temp & Max Temp & Summary \\ \hline
    Monday & 11C & 22C & A clear day with lots of sunshine.  
    However, the strong breeze will bring down the temperatures. \\ \hline
    Tuesday & 9C & 19C & Cloudy with rain, across many northern regions. Clear spells
    across most of Scotland and Northern Ireland,
    but rain reaching the far northwest. \\ \hline
    Wednesday & 10C & 21C & Rain will still linger for the morning.
    Conditions will improve by early afternoon and continue
    throughout the evening. \\
    \hline
    \end{tabular}
\end{center}

\section{Afsnit 3}

\begin{tabular}{|l|l|l|} \hline
\multicolumn{3}{|c|}{Schedulers} \\ \hline
\multirow{3}{*}{Immediate} & RR & Round Robin \\
& EF & Earliest First \\
& LL & Lightest Loaded \\ \hline
\multirow{4}{*}{Batch} & MM & Min-Min \\ 
& MX & Max-Min \\
& DL & Dynamic Level \\ 
& RC & Relative Cost \\ \hline
\multirow{4}{*}{Evolutionary} & PN & This paper \\
& ZO & Genetic Algorithm\\
& TA & Tabu search~\cite{GLOV1986j}\\
& SA & Simlulated Annealing \\ \hline
\end{tabular}

\end{document}

\section{Konklusion}
\begin{center}
\begin{threeparttable}
%\begin{tabular}{c !{\quad}  c !{\quad} ccc @{}c !{\quad} c}
\begin{tabular}{c  c  ccc @{}c  c}
\toprule
                                  &      & \multicolumn{3}{c}{EDU\tnotex{tn:1}} &                       \\
\cmidrule{3-5}
                                  &      & NONE                                 & LOW  & HIGH &  &      \\
\cmidrule{3-5}
\multirow{3}{*}{edu\tnotex{tn:2}} & none & 0,13                                 & 0,19 & 0,06 &  & 0,38 \\
                                  & low  & 0,11                                 & 0,25 & 0,11 &  & 0,47 \\
                                  & high & 0,03                                 & 0,04 & 0,08 &  & 0,15 \\
\midrule
 Total                            &      & 0,27                                 & 0,48 & 0,25 &  & 1    \\
\bottomrule
\end{tabular}
\begin{tablenotes}
  \item[a] \label{tn:1} forklar hvad dette er
  \item[b] \label{tn:2} forklar hvad dette er
\end{tablenotes}
\end{threeparttable}
\end{center}


\end{document}