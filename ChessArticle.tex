\documentclass{article}
%\documentclass[landscape,a4paper,12pt]{article}

\usepackage{graphicx}
\usepackage{amsmath,amsfonts,amssymb,amsthm}
\usepackage{array}
\usepackage{mathtools}
\usepackage[utf8]{inputenc}
\usepackage[danish]{babel}
\usepackage[T1]{fontenc}
\usepackage{fancyhdr}
\usepackage{multicol}
\usepackage{color}
\usepackage{booktabs}
\usepackage{subfig}
\usepackage{skak}
\usepackage{chessboard}
\usepackage{hyperref} %hyperref skal være den sidste pakke



\begin{document}

%\title{Finns Hanham varianter}

%\author{Finn Andersen}
%\date{\today}
%\maketitle

%\begin{abstract}
%Jeg leger lidt med skak-package.
%\end{abstract}

%Bemærk, at \wmove kan give "frie" varianter, men at der skal et mellemrum mellem led
%For eksempel: \wmove{23\ Bb2\ Kg5\ 24\ Nf3\ Qa1}

\setchessboard{smallboard,showmover=false,inverse=true}
\styleC
\begin{multicols}{2}

%\section{Hovedvariant}
\begin{bf}
\begin{Large}
\noindent 1     Hovedvariant
\end{Large}
\end{bf}\\\\
Når man spiller Philidor vil man gerne nå frem til Hanham opstillingen. Og så er det jo rart at vide lidt om hvad der så skal ske...
Hanham opstår efter trækkene:\\ 
\newgame
\mainline{1.e4 d6 2. d4 Nf6 3. Nc3 Nbd7 4. Nf3 e5 5. Bc4 Be7 6. O-O c6}
Bemærk, at jeg spiller \variation{6...c6} og ikke \variation{6...O-O}. Dette giver en mere forceret trækfølge.
\mainline{7. a4 O-O}
\chessboard
\begin{center} Hanham opstilling \end{center}

Hovedvarianten for hvid er nu \mainline{8. Re1} Hvid har en række alternativer: \variation{8. Qe2} behandles i afsnit 2, \variation{8. Ba2} behandles i afsnit 3 og øvrige varianter behandles i afsnit 4.
Efter \variation{8. Re1} spiller jeg den oftest spillede variant \mainline{8...b6} Sort har en række gode muligheder såsom \variation{8...Re8}, \variation{8...Qe7},\variation{8...a5} og \variation{8...exd4}. En interessant mulighed er den såkaldte Pickett shuffle \variation{8...Qe8?! 9. h3 Bd8 10. a5 Bc7 11. d5}. Denne ide er dog mere oplagt i forbindelse med \variation{8...a5}. Trækket \variation{8...b6} har flere ideer. Selvom det ser passiv ud, er det måske det mest aktive træk for sort. Det forbereder en ekspansion på dronningefløjen.\\\\
\begin{bf}
\begin{large}
\noindent 1.1     Hvid spiller 9  h3
\end{large}
\end{bf}\\\\
Hvids træk \mainline{9. h3} er et godt eksempel på de spillemønstre sort skal være opmærksom på. Ideen med hvids træk \bmove{h3} er at forhindre sort i at komme til g4 med springer eller løber. Dette træk er en naturlig forberedelse af \wmove{Be3}. Det er dog muligt i mange varianter at undvære dette træk, da \wmove{Ng4} ikke altid er godt umiddelbart efter \wmove{Be3}. På alle hvids 9. træk undtagen \variation{9. d5} (se afsnit 1.5) kan sort spille hovedideen: \mainline{9...a6}
\chessboard

Sort ønsker at spille \wmove{b5} hvis det er muligt. I nogle tilfælde er sort dog tilfreds med opstillingen med \wmove{b6} og \wmove{a6} da det giver mulighed for at besvare hvids \wmove{d5} med \wmove{c5} og opnå mere plads på dronningefløjen. Denne stilling opstod i partiet Balogh(2662)-Delchev(2629) i 2011 efter en anden trækfølge. Vi følger dette parti videre:
\mainline{10. Ba2} 
Dette træk ser umiddelbart mærkeligt ud. Det er dog tæt knyttet til de hvide planer. Hvid ønsker at spille \wmove{Nh4} efterfulgt af \wmove{Nf5} for at lægge tryk på den sorte kongestilling. Ofte kan dette træk forberedes med \wmove{dxe5}. Hvis sort slår med bonden er det sværere for ham, at finde et godt felt til \wmove{Nd7} som i tilfælde af \wmove{Nh4} gerne vil flytte for at åbne for \wmove{Bc8} der må dække feltet \wmove{f5}. Sort ønsker derfor ofte at slå tilbage på \wmove{e5} med \wmove{Nd7}. Hvis hvids løber nu står på \wmove{c4} må den flytte og sort kan i næste træk bytte på \wmove{f3} hvorved hvids springermanøvre er ødelagt.

Hvis sort spillede \wmove{9... Bb7?!} i stedet for \wmove{9...a6} kunne hvid spille \wmove{10. Ba2!} efterfulgt af \wmove{11. dxe5} og \wmove{12. Nh4} hvorefter sort får svært ved at dække \wmove{f5}. 

En anden pointe med \variation{10.Ba2} som ofte angives i bøger, er at undgå trussel efter sorts \wmove{b5}. Dette er dog mere et spørgsmål om trækfølge og er ikke den vigtigste årsag til trækket. 

Med trækket \wmove{Ba2} sikrer hvid også en rolig plads til løberen hvorfra den kan skyde på \wmove{f7}.\wmove{Ba2} kan også spilles i træk 8 og træk 9 og fører til lignede varianter, se afsnit 3.

Hvid har en række alternativer i denne stilling:\\\\
\begin{bf}Alternativ 1: \variation{10. d5 c5}\end{bf}\\
\variation{10. d5 cxd5?! 11. Nxd5 Bb7} er ikke godt for sort. 
I diagramstillingen efter \wmove{10...c5} vil hvid forsøge at få aktivitet på dronningefløjen med \wmove{b5}. Dette kan forberedes med \wmove{Rb1} men kan også underbygges endnu mere med \wmove{Bf1}, \wmove{Nd2}, \wmove{Nc4}. Sort kan enten støde frem på kongefløjen med \wmove{g6},\wmove{f5} eller prøve at modgå hvids planer på dronningefløjen med \wmove{Ne8},\wmove{Nc7}, \wmove{Rb8} og \wmove{b5}. 
\\
\fenboard{r1bq1rk1/3nbppp/pp1p1n2/2pPp3/P1B1P3/2N2N1P/1PP2PP1/R1BQR1K1 w - - 0 10}
\chessboard

Jeg vælger planen på dronningefløjen og en typisk trækfølge kunne være:\\
\variation{10. d5 c5 11. Bf1 Ne8 12. Nd2 Nc7 13. Nc4 Rb8} eller\\
\variation{10. d5 c5 11. Rb1 Qc7!?} Sort kan lægge pres på c-linien. Måske \wmove{Bb7} og \wmove{Rc8}.\\\\
\begin{bf}Alternativ 2: \variation{10. Be3 Rb8}\end{bf}\\
\fenboard{1rbq1rk1/3nbppp/pppp1n2/4p3/P1BPP3/2N1BN1P/1PP2PP1/R1BQR1K1 w - - 0 10}
\chessboard

Normalt anbefales her 10...\wmove{Bb7}, men som i hovedvarianten bør \wmove{Rb8} også være godt her. Tårnet har forladt dækningen af \wmove{a6}, men sort kan nå at spille \wmove{b5} før hvid kan true \wmove{a6}.\\
For eksempel:\\ \variation{10. Be3 Rb8 11. Qd3 b5 12. axb5 axb5 13. Ba2 b4 14. Ne2}. Den hvide dronning skal stå på \wmove{d3} og ikke \wmove{e2} da springeren skal til \wmove{e2}.\\ Hvid kan forsøge at lægge pres på \wmove{a6} med \wmove{d5}.\wmove{c5} er nu dårligt, da tårnet skal tilbage til \wmove{a8} hvis hvid spiller \wmove{Qe2}. Men varianter som:\\
\variation{10. Be3 Rb8 11. d5 b5! 12. axb5 axb5 13. Bd3 cxd5 14. Nxd5 Nxd5 15. exd5 f5} og\\
\variation{10. Be3 Rb8 11. d5 b5! 12. dxc6 bxc4 13. cxd7 Bxd7} er fine for sort.\\ Også \\
\variation{10. Be3 Rb8 11. d5 b5! 12. axb5 cxb5!? 13. Bd3 Qc7 14. Qe2 Nb6 15. Nd2 Nfd7 16. Na2 f5 17. exf5? Nxd5} er godt for sort.\\\\
\begin{bf}Alternativ 3: \variation{10. Bg5 Bb7}\end{bf}\\
\fenboard{r2q1rk1/1b1nbppp/pppp1n2/4p1B1/P1BPP3/2N2N1P/1PP2PP1/R2QR1K1 w - - 0 10}
\chessboard

Ideen med \wmove{10. Bg5} er at bytte på f6 og herefter spille \wmove{11.d5}. Uden springeren på f6 er d5 svækket for sort og det er ikke muligt, at spille for eksempel:\\  \variation{10. Bg5 Bb7 11. d5 cxd5}, da sort ikke kan få kontrol over d5 efterfølgende og lægge pres på e4. Det er heller ikke godt for sort at spille:\\ \variation{10.Bg5 Rb8 11. d5 c5} da hvid så kan angribe a6 med \wmove{12. Qe2}.
Derfor er det bedste sorte træk \wmove{11...Bb7}.\\Jeg spiller varianten:\\ 
\variation{10. Bg5 Bb7 11. dxe5 Nxe5! 12. Ba2 b5}.\\
\fenboard{r2q1rk1/1b2bppp/p1pp1n2/1p2n1B1/P3P3/2N2N1P/BPP2PP1/R2QR1K1 w - - 0 10}
\chessboard

Kompliceret x er varianten:\\ \variation{10. Bg5 Bb7 11. dxe5 dxe5 12. Nh4 b5 13. Ba2 h6}. Derimod ser varianten:\\ \variation{10. Bg5 Bb7 11. dxe5 dxe5 12. Nh4 g6 13. Bh6 Re8 14. Nf3!} betænkelig ud for sort på grund af truslen \wmove{Ng5}.\\Sort skal passe på ikke at spille b5 for tidligt. Hvid truer hele tiden med \wmove{Bxf6} efterfulgt af \wmove{d5}. For eksempel er denne variant dårlig for sort:\\ \variation{10. Bg5 Bb7 11. Bb3 b5? 12. Bxf6! Bxf6 13. d5!}. Efter \wmove{11. Bb3} bør sort spille \wmove{11...Re8},\wmove{11...Qc7} eller \wmove{11...h6}. Hvis hvid spiller \wmove{d5} kan sort lukke dronningefljøjen med \wmove{c5} og angribe på kongefløjen. Dette giver dog en svaghed på a6. Så sort kunne også besvare \wmove{d5} med \wmove{Qc7} - dette giver en mere levende stilling.

\fenboard{r1bq1rk1/3nbppp/pppp1n2/4p3/P2PP3/2N2N1P/BPP2PP1/R1BQR1K1 b - - 0 10}

Vi går nu tilbage til hovedvarianten efter \wmove{10. Ba2}. I denne situation, hvor hvid ikke længere lægger pres på a6 er det muligt, at spille: 
\mainline{10...Rb8!} 
Herved forbereder sort \wmove{b5} og beholder løberen på c8 hvor den kan dække f5. Andre sorte træk er ikke gode. For eksempel giver \variation{10...b5? 11. axb5 cxb5} hvid en god bondestruktur og et stærkt punkt på d5. Varianten \variation{10...Bb7?! 11. dxe5 dxe5 12. Nh4 Nxe4? 13. Rxe4 Bxh4 14. Rxh4 Qxh4 15. Qd7} giver stor fordel for hvid. 
Efter hovedvarianten er hvids mest logiske træk: 
\mainline{11.Be3}
\chessboard

Hvid har dog en række alternativer:\\
\variation{11. Bg5} giver lige spil, da det bliver svært for hvid at gennemføre manøvren \wmove{Nh4}, \wmove{Nf5} og sort er klar til b5. \\
\variation{11. Qd3?! b5 12. axb5 axb5 13. Ne2 c5 14. c3 exd4! 15.cxd4 c4 16. Qc2 d5} med lige spil. \wmove{11. Qd3} skal dække e4 for at kunne omrokere c3-springeren. \wmove{13. b4} er også en mulighed i denne variant. Bemærk, at tårnet på b8 i modsætning til \wmove{Bb7} dækker b5 og derfor tillader ekspansion med c-bonden.\\
\variation{11. dxe5?! dxe5 12. Nh4 Nc5} er lige spil. Hvis hvid spiller \wmove{Nf5} bliver springeren elimineret.\\ Vi vender nu tilbage til hovedvarianten og partiet  Balogh-Delchev
\mainline{11...b5 12. axb5 axb5 13. b4 Qc7}


\end{multicols}
\end{document}


